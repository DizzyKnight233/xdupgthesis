\chapter{算法示例}
西安电子科技大学是以信息与电子学科为主,
工、理、管、文多学科协调发展的全国重点大学,
直属教育部,是国家“优势学科创新平台”项目和“211工程”项目重点建设高校之一、
国家双创示范基地之一、首批35所示范性软件学院、首批9所示范性微电子学院、
首批9所获批设立集成电路人才培养基地和首批一流网络安全学院建设示范项目的高校之一。
2017年学校信息与通信工程、计算机科学与技术入选国家“双一流”建设学科。
\section{示例1}
学校前身是1931年诞生于江西瑞金的中央军委无线电学校,
是毛泽东等老一辈革命家亲手创建的第一所工程技术学校。
1958年学校迁址西安,1966年转为地方建制,1988年定为现名。
\par
建校90年来,学校始终得到了党和国家的高度重视,
是我国“一五”重点建设的项目之一,
也是1959年中央批准的全国20所重点大学之一。
20世纪60年代,学校就以“西军电”之称蜚声海内外。
毛泽东同志曾先后两次为学校题词:“全心全意为人民服务”、
“艰苦朴素”。
\par
学校现建设有南北两个校区,总占地面积约270公顷,校舍建筑面积130多万平方米。
图书馆馆藏文献约1817万册,其中纸质文献约304万册,电子文献约1513万册,
内容覆盖了学校各个学科或专业。
\par
截至2020年11月底,学校共有全日制在校生36543人,其中本科生22439人,
硕士生11448人,博士生2407人;有在籍网络和函授教育本科生43207人,
网络和函授教育专科生53959人。设有研究生院。
设有通信工程学院、电子工程学院、计算机科学与技术学院(示范性软件学院)、
机电工程学院、物理与光电工程学院、经济与管理学院、数学与统计学院、
人文学院、外国语学院、微电子学院、生命科学技术学院、空间科学与技术学院、
先进材料与纳米科技学院、网络与信息安全学院、马克思主义学院、人工智能学院、
网络与继续教育学院等17个学院。
\par
具体的内容如\algorithmref{alg1}中\lineref{line1}所示。
具体的内容如\algorithmref{alg1}中\lineref{line2}至\lineref{line3}所示。
\begin{algorithm}
\caption{The Bellman-Kalaba algorithm}
\label{alg1}
\begin{algorithmic}[1]
\Procedure {BellmanKalaba}{$G$, $u$, $l$, $p$}
\ForAll {$v \in V(G)$}
\State $l(v) \leftarrow \infty$\label{line1}
\EndFor
\State $l(u) \leftarrow 0$
\Repeat
\For {$i \leftarrow 1, n$}
\State $min \leftarrow l(v_i)$
\For {$j \leftarrow 1, n$}\label{line2}
\If {$min > e(v_i, v_j) + l(v_j)$}
\State $min \leftarrow e(v_i, v_j) + l(v_j)$
\State $p(i) \leftarrow v_j$
\EndIf
\EndFor\label{line3}
\State $l'(i) \leftarrow min$
\EndFor
\State $changed \leftarrow l \not= l'$
\State $l \leftarrow l'$
\Until{$\neg changed$}
\EndProcedure
\Statex
\Procedure {FindPathBK}{$v$, $u$, $p$}
\If {$v = u$}
\State \textbf{Write} $v$
\Else
\State $w \leftarrow v$
\While {$w \not= u$}
\State \textbf{Write} $w$
\State $w \leftarrow p(w)$
\EndWhile
\EndIf
\EndProcedure
\end{algorithmic}
\end{algorithm}
\par
学校是国内最早建立信息论、信息系统工程、雷达、微波天线、电子机械、
电子对抗等专业的高校之一,开辟了我国IT学科的先河,
形成了鲜明的电子与信息学科特色与优势。
“十三五”期间,学校获批8个国防特色学科。
学校现有2个国家“双一流”重点建设学科群(包含信息与通信工程、电子科学与技术、
计算机科学与技术、网络空间安全、控制科学与工程5个一级学科),
2个国家一级重点学科(覆盖6个二级学科),1个国家二级重点学科,34个省部级重点学科,
14个博士学位授权一级学科,26个硕士学位授权一级学科,10个博士后科研流动站,
65个本科专业。全国第四轮一级学科评估结果中,3个学科获评A类:
电子科学与技术学科评估结果为A+档,并列全国第1;
信息与通信工程学科位于A档;计算机科学与技术学科评估结果为A-档,
学校电子信息类学科继续保持国内领先水平。
根据ESI公布数据,学校工程学和计算机科学均位列全球排名前1‰。
\par
学校树立了以人为本、教师是大学核心竞争力的理念,
锻造了一支结构合理、富有创新精神的教师队伍。
现有专任教师2300余名,其中,博士生导师700余人,硕士生导师1500余人。
学校有院士3人,“万人计划”入选者28人,长江学者36人,
国家自然科学基金创新研究群体2个,科技部重点创新团队5个,
教育部创新团队6个,国家级教学名师4人,国家级教学团队6个,973项目首席科学家3人,
教育部新世纪优秀人才51人,“何梁何利”科学与技术奖获得者4人,
教育部教学指导委员会委员19人,享受政府特殊津贴165人。
\par
学校不断地创新教育理念,深化教学内容、课程体系与实践教学改革,
大力推进素质教育,取得了显着成果。现有国家级特色专业14个,
国家级精品课程13门,国家级精品资源共享课11门,
国家级视频公开课3门,国家精品在线开放课程9门,国家级一流本科课程13门,
建设有3个国家人才培养及教学基地、6个国家级实验教学示范中心、
3个国家级虚拟仿真实验中心,以及3个国家级人才培养模式创新实验区。
学校人才培养素以理论基础扎实、工程实践能力突出、
创新意识强等特色在全国高校中形成了“品牌”。
学校坚持“因材施教、分类培养”的教育理念,
积极探索实施“卓越工程师教育培养计划”、
“钱学森空间科学实验班”和“科教结合协同育人行动计划”等一系列创新型人才培养模式改革。
近五年来,学校本科生参与课外科技活动的普及率高,
获得各类省级、国家级学科和科技竞赛奖3600余项。
研究生和本科毕业生总体就业率一直保持在96\%以上,位居全国高校前列。
2006年,学校顺利通过教育部本科教学工作水平评估并获得“优秀”;
2020年,学校获中国高校“就业最佳典范奖”。
\section{示例2}
多年来,学校致力于电子信息技术领域的系统研制、科技攻关、工程研发等,
创造了我国电子与信息技术领域等多项第一,
包括第一台气象雷达、第一套流星余迹通讯系统、第一台可编程雷达信号处理机、
第一台毫米波通讯机,以及我军通信装备史上第一部“塞绳电报互换机”、
第一台“塔型管空腔振荡器”、第一套“三坐标相控阵雷达”等,
为我国信息化、国防现代化做出了重要的贡献。
学校现有9个国家级科技创新基地、1个科工局科技创新基地,10个教育部科技创新基地、
29个陕西省科技创新基地,2013年入选国家级创新人才培养示范基地。
先后牵头承担了“973”、“863”、重大专项、国家重点研发计划、国家自然科学重大项目、
国家重大项目科研仪器研制项目等重大、重点项目,产生了一批标志性的研究成果。
2013年以来,学校科研指标稳步提升,在认知雷达、移动通讯、网络信息安全、
高功率微波集成器件、智能计算、大型天线机电耦合等方面取得了卓有成效的成果,
2012年以来学校获国家科技奖励21项。
2014年,学校牵头的“信息感知技术协同创新中心”通过国家“2011计划”认定,
位列行业产业类第一,进一步奠定了学校在全国高校中突出的国防科研特色优势地位。
\begin{algorithm}
\caption{Euclid's algorithm}\label{euclid}
\begin{algorithmic}[1]
\Procedure{Euclid}{$a,b$}\Comment{The g.c.d. of a and b}
\State $r\gets a\bmod b$
\While{$r\not=0$}\Comment{We have the answer if r is 0}
\State $a\gets b$
\State $b\gets r$
\State $r\gets a\bmod b$
\EndWhile\label{euclidendwhile}
\State \textbf{return} $b$\Comment{The gcd is b}
\EndProcedure
\end{algorithmic}
\end{algorithm}
\par
学校大力加强产学研相结合,不断增强科技创新能力。
建设有中国西部军民融合创新谷暨西安电子谷、陕西工业研究院、国家大学科技园,
同时与国内大型知名企事业单位联合建立股份制公司,
成立战略联盟、设立企业基金、建立联合实验室及研究生实习基地,
有力促进了科技成果的转化。
学校积极开展国际国内的交流与合作,拓展外部发展空间。
学校先后与35个国家和地区的155所大学及研究机构建立友好关系,
与10余个研究所、研究中心、企业集团建立了长期战略合作伙伴关系,
与西安、广州、青岛、重庆等地方政府开展深入合作,
共建研究院所、研究中心、新型研发机构,
与跨国公司建立66个联合实验室,
基本形成多方位、多层次、宽领域的对外合作创新发展格局。
\par
建校90年来,学校先后为国家输送了31万余名电子信息领域的高级人才,
产生了120多位解放军将领,成长起了24位院士
(1977年恢复高考以后院士校友20位,位列全国前茅),
10余位国家副部级以上领导,培养了联想创始人柳传志,
国际GSM奖获得者李默芳,欧洲科学院院士、著名的纳米技术专家王中林,
“天宫一号”目标飞行器总设计师杨宏等一大批IT行业领军人物和技术骨干、
科研院所所长和大学校长等,为国家建设和社会进步做出了重要贡献。
\par
在全面建设社会主义现代化国家新征程中,
西安电子科技大学将继续坚持走内涵式发展道路,
秉承“全心全意为人民服务”的办学宗旨,
坚持“立足西部、育人育才、强军拓民、服务引领、团结实干”的发展思路,
坚持立德树人根本任务,全面提升教育质量,
为把学校建设成为电子信息特色鲜明的一流大学而不懈奋斗!
