\chapter{研究生学位论文的编辑、打印、装订要求}
\section{学位论文封面的编辑和打印要求}
学位论文的封面由研究生院按国家规定统一制定印刷,封面内容必须打印,不得手写。
\section{学位论文的版面设置要求}
(1)行间距:固定值20磅(题名页除外)。
\par
(2)字符间距:标准。
\par
(3)页眉设置:单面页码页眉标题为章节题目,每一章节的起始页必须在单面页码,双面页码页眉标题统一为“西安电子科技大学博/硕士学位论文”,页眉标题居中排列,字体为宋体,字号为五号。页眉文字下添加双横线,双横线宽度为0.5 磅,距正文距离为:上下各1磅,左右各4磅。
\par
(4)页码设置:学位论文的前置部分和主体部分分开设置页码,前置部分的页码用罗马数字标识,字体为Times New Roman,字号为小五号;主体部分的页码用阿拉伯数字标识,字体为宋体,字号为小五号。页码统一居于页面底端中部,不加任何修饰。
\par
(5)页面设置:为了便于装订,要求每页纸的四周留有足够的空白边缘,其中页边距为上3厘米、下2厘米;内侧2.5厘米、外侧2.5厘米;装订线为0.5厘米;页眉2厘米,页脚1.75厘米。
\section{学位论文的打印、装订要求}
(1)打印:学位论文必须用A4纸页面排版,双面打印;
\par
(2)装订:依次按照中文题名页、英文题名页、声明、摘要、插图索引、表格索引、符号对照表、缩略语对照表、目录、正文、附录(可选)、参考文献、致谢、作者简介的顺序,用学校统一印制的学位论文封面装订成册。盲审论文必须删除致谢部分的文字内容(致谢标题须保留)以及封面和研究成果中的作者和指导教师姓名,研究成果列表中应体现作者的排序,如第一作者、第一发明人等。
\section{其他说明}
本规定由研究生院负责解释,从申请2015年9月毕业和授位的研究生开始执行,其它有关规定同时废止。研究生毕业论文撰写要求参照学位论文撰写要求执行。
